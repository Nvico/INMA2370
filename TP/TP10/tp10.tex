
\subsection*{Exercice 10.1 A balloon}\\
%%% AUTHOR: Florentin GOYENS %%%

\textbf{1. Meaning of the equations:}\\

We are given the following system
$$\begin{array}{lll}
\dot{\theta} & = & -\frac{\theta}{\tau_{1}}+u\\
\dot{v} & = & -\frac{v}{\tau_{2}}+\sigma\theta \\
\dot{h} & = & v.\\
\end{array}$$

The meaning of the equations in their respective order is
\begin{itemize}
\item Heat conservation,
\item Newton's law,
\item Definition of velocity.\\\end{itemize}


\textbf{2. Controllability:}\\

We now analyse the controllability of the system. A theorem states that the system is controllable if and only if 
$$\text{rank}[B \text{  } AB\text{  } A^{2}B \text{  }\dots\text{  } A^{n-1}B]=n$$
where the matrix $A$ and $B$ are defined from the general form of a system of $n$ variables
$$\dot{x}=Ax+Bu.$$

In this case we have $B=\begin{pmatrix}
1\\
0\\
0\\
\end{pmatrix}$ and $A=\begin{pmatrix}
\frac{-1}{\tau_{1}} & 0 & 0\\
\sigma & \frac{-1}{\tau_{2}} & 0 \\
0 & 1 & 0 \\
\end{pmatrix}.$

Straightforward computation gives the matrix of controllability $M=\begin{pmatrix}
1      & \frac{-1}{\tau_{1}} & \frac{1}{\tau_{1}^{2}}\\
0 & \sigma & \textcolor{blue}{-\frac{\sigma}{\tau_{1}}-\frac{\sigma}{\tau_{2}}} \\
0      & 0 & \sigma \\
\end{pmatrix}.$

Clearly the matrix has full rank since we can assume $\sigma>0$ for physical reasons.
\textcolor{blue}{Remark : Instead of $\frac{-2\sigma}{\tau_{1}}$ (as in the current solution) , the coefficient $M_{23}$ should be $-\frac{\sigma}{\tau_{1}}-\frac{\sigma}{\tau_{2}}$}.\\


\textbf{3. Trajectory control:}\\

We look for a transformation to express the system under a Brunovski form. 



Starting from $\dot{h}=v$, we set $z_{1}=h$ and $z_{2}=v$ wich already gives $\dot{z_{1}}=z_{2}$. Looking for $z_{3}$ we write $z_{3}=\dot{z_{2}}=\dot{v}=\sigma\theta-\frac{v}{\tau_{2}}$.



This allows us to fully define our transformation $(z_{1},z_{2},z_{3})=T(\theta,v,h)$ by
$$\begin{array}{lll}
z_{1}=h\\
z_{2}=v\\
z_{3}=\sigma\theta-\frac{v}{\tau_{2}}.\\
\end{array}$$

We find 
\begin{align*}
\dot{z_{3}}&=\sigma\dot{\theta}-\frac{\dot{v}}{\tau_{2}}\\
&=\sigma\dot{\theta}-\frac{z_{3}}{\tau_{2}}\\
&=\sigma(-\frac{\theta }{\tau_{1}}+u)-\frac{z_{3}}{\tau_{2}}\\
&=\sigma(-\frac{1}{\tau_{1}\sigma}(z_{3}+\frac{z_{2}}{\tau_{2}})+u)-\frac{z_{3}}{\tau_{2}}\\
&=-\frac{z_{2}}{\tau_{1}\tau_{2}}-z_{3}(\frac{1}{\tau_{1}}+\frac{1}{\tau_{2}})+\sigma u\\
\end{align*}

We want to plan a trajectory from $z(t=0)=T(x(0))=T\begin{pmatrix}
0\\
0\\
h_{0}\\
\end{pmatrix}=\begin{pmatrix}
h_{0}\\
0\\
0\\
\end{pmatrix}$ to $z(t=1)=T(x(1))=T\begin{pmatrix}
0\\
0\\
h_{1}\\
\end{pmatrix}=\begin{pmatrix}
h_{1}\\
0\\
0\\
\end{pmatrix}.$

Having $6$ conditions we set a trajectory for $z_{1}$ that is a polynomial of degree 5, 
$$z_{1}(t)=\lambda_{5} t^{5}+\lambda_{4} t^{4} + \lambda_{3} t^{3} + \lambda_{2} t^{2} +\lambda_{1} t+\lambda_{0}.$$
By differentiating we find
$$z_{2}(t)=\dot{z_{1}(t)}=5\lambda_{5} t^{4}+4\lambda_{4} t^{3} + 3\lambda_{3} t^{2} + 2\lambda_{2} t +\lambda_{1},$$
$$z_{3}(t)=\dot{z_{2}(t)}=20\lambda_{5} t^{3}+12\lambda_{4} t^{2} + 6\lambda_{3} t + 2\lambda_{2}.$$

We find the unknown coefficients solving

$$\left\{\begin{array}{ccc}
z_{1}(0) & = & h_{0}\\
z_{1}(1) & = & h_{1}\\
z_{2}(0) & = & 0\\
z_{2}(1) & = & 0\\
z_{3}(0) & = & 0\\
z_{3}(1) & = & 0\\
\end{array}\right.$$

This gives 
$$\left\{\begin{array}{ccc}
\lambda_{0} & = & h_{0}\\
\lambda_{1} & = & 0\\
\lambda_{2} & = & 0\\
\lambda_{3}& = & 10(h_{1}-h_{0})\\
\lambda_{4}& = & -15(h_{1}-h_{0})\\
\lambda_{5}& = & 6(h_{1}-h_{0})\\
\end{array}\right.$$



This gives us the trajectories
\begin{align*}
z_{1}(t)&=6(h_{1}-h_{0}) t^{5}-15(h_{1}-h_{0}) t^{4} + 10(h_{1}-h_{0}) t^{3} + h_{0},\\
z_{2}(t)&=30(h_{1}-h_{0}) t^{4}-60(h_{1}-h_{0}) t^{3} + 30(h_{1}-h_{0}) t^{2},\\
z_{3}(t)&=120(h_{1}-h_{0}) t^{3}-180(h_{1}-h_{0}) t^{2} + 60(h_{1}-h_{0}) t.
\end{align*}


We can get $u(t)$ by the formula
$$u(t)=\dfrac{\dot{z_{3}(t)}-\alpha(z(t))}{\beta},$$
where $\alpha(z(t))=-\frac{z_{2}(t)}{\tau_{1}\tau_{2}}-z_{3}(t)(\frac{1}{\tau_{1}}+\frac{1}{\tau_{2}})$ and $\beta=\sigma$.\\


\subsection*{Exercice 10.2 A biochemical reactor}
%%% AUTHOR: Florentin GOYENS %%%
The velocity of the reaction is $r(x_{A},x_{B})=kx_{A}x_{B}$. The volume being constant, we can set $F^{in}=F^{out}=d$. It follows, 

$$\left\lbrace\begin{array}{ccl}
\dot{x_{A}} & = &  -r+d(x_{A}^{in}-x_{A}) \\
\dot{x_{B}} & = & r-dx_{B}  \\
\dot{x_{C}} & = & r-dx_{C}.  \\
\end{array}\right.$$

When $u=d$, we show that the system is controlable noticing that matrix $[B\hspace*{0.3cm} AB\hspace*{0.3cm} A^{2}B]$ has full rank\textcolor{blue}{Il faudrait confirmer.}.

In the second case we have $u=x_{A}^{in}$ and we try to express the variables that are not controlable.
We define the linear change of variable $z=Tx$ that corresponds to 

$$\left\lbrace\begin{array}{ccl}
z_{1} & = & x_{A} \\
z_{2} & = & x_{B} \\
z_{3} & = & x_{B}-x_{C}. \\
\end{array}\right.$$

The inverse is naturally 

$$\left\lbrace\begin{array}{ccl}
x_{A} & = & z_{1} \\
x_{B} & = & z_{2} \\
x_{C} & = & z_{2}-z_{3}. \\
\end{array}\right.$$

We now use $\dot{z}=T\dot{x}=\left.Tf(x,u)\right]_{x=T^{-1}z}$. This yields 
$$\dot{z}=\begin{pmatrix}
-kz_{1}z_{2}+d(u-z_{1})\\
kz_{1}z_{2}-dz_{2}\\
-dz_{3}.\\
\end{pmatrix}$$

This shows that the state $z_{3}$ is not commandable. Since it is not influenced by the input $u$ (directly or undireclty). Note that $z_{3}$ coresponds to the difference of concentration between components $B$ and $C$.



\subsection*{Exercice 10.3: A satelite}
%%% AUTHOR: Florentin GOYENS %%%
We will compute the rank of contralability.
 Starting from $g_{1}(x)=(b_{1},0,0)$ and $g_{2}(x)=(0,b_{2},0)$, we find $g_{3}(x)=[f,g_{1}]=(0,-b_{1}a_{2}x_{3},b_{1}a_{3}x_{2})$. 
 We have that the dimension of $span\{g_{1},g_{2},g_{3}\}=3$ and therefore the system is controlable. 

\subsection*{Exercice 10.4: A scuba diver}
%%% AUTHOR: Florentin GOYENS %%%
We set $x_{1}=h$, $x_{2}=\dot{h}$ and $x_{3}=q$ to obtain the following equivalent system
$$\left\lbrace\begin{array}{ccccl}
\dot{x_{1}} & = & g_{1} & = & x_{2}\\
M\dot{x_{2}} & = & Mg_{2} & = & Mg-\rho g(V_{0}+x_{3}\dfrac{RT}{P_{0}+\rho h})\\
\dot{x_{3}} & = & g_{3} & = & u\\
\end{array}\right.$$


1. We show that the system can be transformed in a Brunoski form with a straightforward application of lemma 6.14 since the system is already triangular. We see that $\frac{\partial g_{1}}{\partial x_{2}}\neq 0$, $\frac{\partial g_{2}}{\partial x_{3}}\neq 0$ and $\frac{\partial g_{3}}{\partial u}\neq 0$.

2. The system is commandable because for linear systems, we have the equivalence between commandability and the existence of a Brunovki form (theorem 10.12).

3. We find a transformation that puts the system in a Brunovki form. With 

$$\left\lbrace\begin{array}{ccl}
z_{1} & = & x_{1}\\
z_{2} & = & x_{2}\\
z_{3} & = & g-\frac{\rho g}{M}(V_{0}+x_{3}\dfrac{RT}{P_{0}+\rho h})\\
\end{array}\right.$$

we get 
$$\left\lbrace\begin{array}{ccl}
\dot{z_{1}} & = & z_{2}\\
\dot{z_{2}} & = & z_{3}\\
\dot{z_{3}} & = & -\frac{\rho h}{M}\dfrac{RT}{P_{0}+\rho h}u.\\
\end{array}\right.$$

We assume from the context that we want to start and end our trajectory with a speed of zero. This gives us $4$ conditions 
$$\left\lbrace\begin{array}{ccl}
z_{1}(0) & = & \bar{h_{1}}\\
z_{1}(t_{f}) & = & \bar{h_{2}}\\
\dot{z_{1}}(0) & = & 0\\
\dot{z_{1}}(t_{f}) & = & 0\\
\end{array}\right.$$

As we set $z_{1}(t)=\lambda_{3}(\frac{t}{t_{f}})^{3}+\lambda_{2}(\frac{t}{t_{f}})^{2}+\lambda_{1}(\frac{t}{t_{f}})+\lambda_{0}$, we are going to have $4$ equations to find the unknown $\lambda_{i}$. This will give us a expression for $z_{1}(t)$. From wich we will deduce $(z_{3})'=(z_{2})''=(z_{1})'''$. And of course $u$ comes from $\dot{z_{3}}  =  -\frac{\rho h}{M}\dfrac{RT}{P_{0}+\rho h}u$.



\subsection*{Exercice 10.5 A jumping robot}
%%% AUTHOR: Florentin GOYENS %%%

\textbf{1. Model:}\\

Given the equations 
$$\left\{ \begin{array}{rrr}
M\dot{\theta}+ m(z+d)^{2}(\dot{\theta}+\dot{\phi}) & = & 0\\
u_{1} & = & \dot{\phi}\\
u_{2} & = & \dot{z}\\
\end{array}\right.$$
One can find $$\dot{\theta}=\dfrac{-m(z+d)^{2}}{M+m(z+d)^{2}}u_{1}.$$
And therefore express the system of variables $z, \theta, \phi$ with entrance $u_{1}, u_{2}$
$$\begin{pmatrix}
\dot{z}\\
\dot{\theta}\\
\dot{\phi}\\
\end{pmatrix}=
\underbrace{
\begin{pmatrix}
0\\
\dfrac{-m(z+d)^{2}}{M+m(z+d)^{2}}=k(z)\\
1\\
\end{pmatrix}}_{g_{1}}u_{1}+
\underbrace{
\begin{pmatrix}
1\\
0\\
0\\
\end{pmatrix}}_{g_{2}}u_{2}.$$

\textbf{2. Controllability:}\\

We compute
$$[g_{1},g_{2}]=\dfrac{\partial g_{2}}{\partial x}g_{1}-\dfrac{\partial g_{1}(z)}{\partial z}g_{2}=\begin{pmatrix}
0\\
k'(z)\\
0\\
\end{pmatrix}.$$

As we see, $[g_{1},g_{2}]\notin \text{span}\{g_{1},g_{2}\}\text{, } \forall z>-d$. The rank of reachability is $3$ and according to theorem $10.10$ we have controllability for $z>-d$.

