% AUTHOR : HADRIEN VAN LIERDE
\subsection*{Exercise 9.1 : RLC circuit}
Given the following RLC circuit\\ 

\textcolor{blue}{Add image of the RLC circuit}~\\

The characteristic curve of the relation between current and tension of the non linear resistance is denoted by $i=g(v)$ and is a monotonically increasing function as shown on the following graph :\\

\textcolor{blue}{Add image of the characteristic curve}~\\

\begin{enumerate}
\item Give a state-space representation of the system,
\item Compute the equilibrium points of the system,
\item Using energy as a Lyapunov function, analyse the global stability of the equilibriums using the second method of Lyapunov.
\end{enumerate}

\textbf{Answer: }~\\

Denoting by $V_c$ the tension of the capacity and $I_l$ the current in the inductance, we obtain the following equations using Kirchoff laws:
$$
\begin{array}{l}
u-g(V_c)-i_L-C\frac{dV_c}{dt}=0\\
L\frac{dI_l}{dt}-V_c=0
\end{array}
$$

Using the notations $V_c=x_1$ and $I_l=x_2$, we obtain the following state-space representation:
$$
\begin{array}{rcl}
C\dot{x}_1&=&u-g(x_1)-x_2\\
L\dot{x}_2&=&x_1
\end{array}
$$

The equilibrium points $(\bar{x}_1,\bar{x}_2,\bar{u})$ are found by solving the following equations:
$$
\begin{array}{l}
\bar{x}_1=0\\
\bar{u}-g(\bar{x}_1)-\bar{x}_2=0
\end{array}
$$
which gives $\bar{x}_1=0$ and $\bar{x}_2=\bar{u}$.\\

One can use $V(x_1,x_2)=\frac{1}{2}L(x_2-u)^2+\frac{1}{2}Cx_1^2$ as a Lyapunov function. We verify the assumptions of La Salle principle on the whole domain $\mathbb{R}^2$:
\begin{enumerate}
\item $V(\bar{x}_1,\bar{x}_2)=0<V(x_1,x_2)$ whenever $(x_1,x_2)\neq (\bar{x}_1,\bar{x}_2)$,
\item Derivative:
$$
\begin{array}{rcl}
\dot{V}(\bar{x}_1,\bar{x}_2)&=& L\dot{x}_2(x_2-u)+C\dot{x}_1x_1\\
&=& -x_1g(x_1)\leq 0\text{ }\forall x_1\in\mathbb{R}
\end{array}
$$

\item We verify that
$$
\begin{array}{rcl}
C\dot{x}_1&=&\bar{u}-g(x_1)-x_2\\
L\dot{x}_2&=&x_1\\
\dot{V}(\bar{x}_1,\bar{x}_2) &=& -x_1g(x_1)=0
\end{array}
$$
implies $x_1=0=\bar{x}_1$ and $x_2=\bar{u}$.
\end{enumerate}
As the assumptions of La Salle principle are verified, we conclude that the equilibrium is asymptotically stable.

\subsection*{Exercise 9.2 : Chemical Reactor}

Given a CSTR chemical reactor where the following reaction is taking place:
$$
X_1 + X_2 \rightarrow 2X_2
$$
\begin{enumerate}
\item establish a state-space model of the system based on the following assumptions:
\begin{itemize}
\item the reaction kinetics is given by the law of mass action,
\item the reactor is only supplied with $X_1$ ($x_1^{in}=cst$),
\item the flow rate is the input,
\end{itemize}

\item show that this system has an equilibrium in the positive orthant,
\item show that the function $V=x_1-\bar{x}_1\ln(x_1)+x_2-\bar{x}_2\ln(x_2)$ is a Lyapunov function in the positive orthant,
\item prove that the equilibrium is globally asymptotically stable in the positive orthant.
\end{enumerate}

\textbf{Answer: }~\\

\paragraph{State-Space model}~\\

We denote by $x_1$, $x_2$ and $x_3$ the concentrations of each product. The reaction kinetics is given by the formula $r(x_1,x_2)=-kx_1x_2$. As the volume is constant ($F_{in}=F_{out}$ and $V$ is constant), one can write the following differential equations:
$$
\begin{array}{rcl}
\frac{dx_1}{dt} &=& -kx_1x_2+\frac{F_{in}}{V}x_1^{in}-\frac{F_{in}}{V}x_1\\
\frac{dx_2}{dt} &=& kx_1x_2-\frac{F_{in}}{V}x_2
\end{array}
$$
Denoting the input as $u=\frac{F_{in}}{V}$, one obtains the state space model
$$
\begin{array}{rcl}
\dot{x}_1&=&-kx_1x_2+u(x_1^{in}-x_1)\\
\dot{x}_2&=&kx_1x_2-ux_2
\end{array}
$$

\paragraph{Equilibrium}~\\
To find an equilibrium $(\bar{x}_1,\bar{x}_2,\bar{u})$, we solve the following system:
$$
\begin{array}{l}
-k\bar{x}_1\bar{x}_2+\bar{u}(_1^{in}-\bar{x}_1)=0\\
k\bar{x}_1\bar{x}_2-\bar{u}\bar{x}_2=0
\end{array}
$$
which gives two equilibrium points:
\begin{enumerate}
\item $(\bar{x}_1,\bar{x}_2,\bar{u})=(x_1^{in},0,\bar{u})$ for $\bar{u}\leq 0$,
\item $(\bar{x}_1,\bar{x}_2,\bar{u})=(\frac{\bar{u}}{k},x_1^{in}-\frac{\bar{u}}{k},\bar{u})$ for $0<\frac{\bar{u}}{k}<x_1^{in}$.
\end{enumerate}

\paragraph{Lyapunov function}~\\

We focus on the equilibrium $(\bar{x}_1,\bar{x}_2,\bar{u})=(\frac{\bar{u}}{k},x_1^{in}-\frac{\bar{u}}{k},\bar{u})$ for $0<\frac{\bar{u}}{k}<x_1^{in}$ and show that $V=x_1-\bar{x}_1\ln(x_1)+x_2-\bar{x}_2\ln(x_2)$ is a Lyapunov function for this equilibrium in the positive orthant.

\begin{enumerate}
\item $V(\bar{x}_1,\bar{x}_2)=0<V(x_1,x_2)$ whenever $(x_1,x_2)\neq (\bar{x}_1,\bar{x}_2)$:
Let us minimize $V(x_1,x_2)$, setting the derivatives of $V$ to zero, we obtain the system
$$
\begin{array}{rcl}
\frac{\partial V}{\partial x_1} &=& 1-\frac{\bar{u}}{kx_1}=0\\
\frac{\partial V}{\partial x21} &=& 1-\frac{1}{x_2}(x_1^{in}-\frac{\bar{u}}{k})=0
\end{array}
$$
These equations are verified if and only if $(x_1,x_2)=(\bar{x}_1,\bar{x}_2)$, moreover,
$$
\nabla^2V=\begin{bmatrix}
\frac{\bar{u}}{kx_1^2} & 0\\
0 & \frac{1}{x_2^2}(x_1^{in}-\frac{\bar{u}}{k})
\end{bmatrix}
$$
which is positive definite if $0<\frac{\bar{u}}{k}<x_1^{in}$. This guarantees that $(\bar{x}_1,\bar{x}_2)$ is the unique global minimizer of $V$.

\item Derivative: after some calculations, one can show that
$$
\begin{array}{rcl}
\dot{V}(x_1,x_2) &=& \dot{x_1}(1-\frac{\bar{x}_1}{x_1})+\dot{x_2}(1-\frac{\bar{x}_2}{x_2})\\
&=& 2\bar{u}x_1^{in}-\frac{\bar{u}^2x_1^{in}}{kx_1}-kx_1^{in}x_1\\
&=& -\frac{x_1^{in}}{k}(\frac{\bar{u}^2}{x_1}+k^2x_1-2k\bar{u})\\
&=& -\frac{c}{k}(\frac{\bar{u}}{\sqrt{x_1}}-k\sqrt{x_1})^2\leq 0
\end{array}
$$
whenever $(x_1,x_2)$ is taken in the positive orthant.

\item We verify that :
$$
\begin{array}{rcl}
\dot{x}_1&=&-kx_1x_2+u(x_1^{in}-x_1)\\
\dot{x}_2&=&kx_1x_2-ux_2\\
\dot{V}(x_1,x_2) &=& -\frac{c}{k}(\frac{\bar{u}}{\sqrt{x_1}}-k\sqrt(x_1))^2 = 0
\end{array}
$$
implies $(x_1,x_2)=(\bar{x}_1,\bar{x}_2)=(\frac{\bar{u}}{k},x_1^{in}-\frac{\bar{u}}{k})$
\end{enumerate}

\paragraph{Asymptotic stability :}~\\
As the assumptions of La Salle principle are verified, we conclude that the equilibrium $(\bar{x}_1,\bar{x}_2,\bar{u})=(\frac{\bar{u}}{k},x_1^{in}-\frac{\bar{u}}{k},\bar{u})$ is globally asymptotically stable in the positive orthant if $0<\frac{\bar{u}}{k}<x_1^{in}$.

\subsection*{Exercise 9.3 : Mechanical system}
We consider a mechanical system with one degree of freedom. The position variable is denoted by $x_1$. A conservative force (due to a potential) is applied to the system along with a viscous linear friction. The potential energy is given by:
$$
E_p(x_1)=\int_0^{x_1}\frac{\sigma}{K+|\sigma|}d\sigma
$$
\begin{enumerate}
\item Establish a state-space model of the system,
\item Compute the equilibrium points,
\item Analyse the stability of the equilibrium points using the first (direct) method of Lyapunov.
\end{enumerate}

\textbf{Answer: }~\\

\paragraph{State-space model}
We denote by $u$ the external force applied to the system. The dynamic of the system is described by the following differential equation:
$$
m\ddot{x_1}=-\frac{x_1}{K+|x_1|}-k\dot{x}_1+u
$$
which gives the following state-space model:
$$
\begin{array}{rcl}
\dot{x_1} &=& x_2\\
m\dot{x_2} &=& -\frac{x_1}{K+|x_1|}-kx_2+u
\end{array}
$$

\paragraph{Equilibrium points}
The equilibrium points are found by solving the system:
$$
\begin{array}{rcl}
\bar{x}_2 &=& 0\\
\frac{\bar{x}_1}{K+|\bar{x}_1|}+k\bar{x}_2 &=& \bar{u}
\end{array}
$$
The two following cases arize:
\begin{enumerate}
\item case 1: $0\leq\bar{u}<1$ then $\bar{x}_1=\frac{\bar{u}K}{1-\bar{u}}$
\item case 2: $-1<\bar{u}<0$ then $\bar{x}_1=\frac{\bar{u}K}{1+\bar{u}}$
\end{enumerate}
We have thus an equilibrium point whenever $-1<\bar{u}<1$.

\paragraph{Stability of equilibrium points}
We choose the following Lyapunov function:
$$V(x_1,x_2)=\frac{1}{2}mx_2^2+\int_0^{x_1}\frac{\sigma}{K+|\sigma|}d\sigma-x_1\bar{u}$$
We verify the assumptions of the La Salle principle in $\mathbb{R}^2$
\begin{enumerate}
\item $V(\bar{x}_1,\bar{x}_2)=0<V(x_1,x_2)$ whenever $(x_1,x_2)\neq (\bar{x}_1,\bar{x}_2)$: let us minimize $V$: $V$ is clearly differentiable and its gradient is zero if
$$
\begin{array}{rcl}
x_2 &=& 0\\
\frac{x_1}{K+|x_1|}+kx_2 &=& \bar{u}
\end{array}
$$
which is verified only if $(x_1,x_2)=(\bar{x}_1,\bar{x}_2$ and $-1<\bar{u}<1$.
Moreover, if $x_1>0$,
$$
\nabla^2V=\begin{bmatrix}
\frac{K}{(K+x_1)^2} & 0\\
0 & 1\\
\end{bmatrix}
$$
which is positive definite and if $x_1<0$,
$$
\nabla^2V=\begin{bmatrix}
\frac{K}{(K-x_1)^2} & 0\\
0 & 1\\
\end{bmatrix}
$$
which is also positive definite. As we verify,
$$
\lim_{x_1\rightarrow 0^+}\nabla^2V=\lim_{x_1\rightarrow 0^-}\nabla^2V
$$
$V$ is two times differentiable and its Hessian matrix is positive definite on for all $(x_1,x_2)\in\mathbb{R}^2$. This guarantees that $(\bar{x}_1,\bar{x}_2)$ is the unique global minimizer of $V$.
\item Derivative:
$\dot{V}(x_1,x_2)=mx_2\dot{x}_2+\frac{x_1}{K+|x_1|}x_2-\bar{u}x_2=-kx_2^2\leq 0$
\item We verify 
$$
\begin{array}{rcl}
\dot{x_1} &=& x_2\\
m\dot{x_2} &=& -\frac{x_1}{K+|x_1|}-kx_2+u\\
\dot{V}(x_1,x_2) &=& -kx_2^2=0
\end{array}
$$
only if $(x_1,x_2)=(\bar{x}_1,\bar{x}_2)$ when $-1<\bar{u}<1$.
\end{enumerate}

As the assumptions of the La Salle principle are verified, $(\bar{x}_1,\bar{x}_2)$ is globally asymptotically stable when $-1<\bar{u}<1$.

\subsection*{Exercise 9.4 : Model of a neurone}
The model of Naka-Rushton describing the dynamics of a neurone in the short term memory follows the state-space model:
$$
\begin{array}{rcl}
\dot{x}_1 &=& -x_1 +\frac{ux_2}{1+x_2}\\
\dot{x}_2 &=& -x_2 +\frac{ux_1}{1+x_1}
\end{array}
$$

\begin{enumerate}
\item Show that the system is positive,
\item Analyse the existence and stability of equilibriums in the positive orthant (first method of Lyapunov),
\item For a constant input $\bar{u}$ such that $0<\bar{u}<1$, show that all trajectories in the positive orthant converge to the origin. Use the Lyapunov function $V(x_1,x_2)=\frac{1}{2}(x_1^2+x_2^2)$.
\end{enumerate}

\textbf{Answer: }~\\

\paragraph{Positiveness of the system: }~\\
We verify the two following condition:
\begin{enumerate}
\item when writing the system in the form $\dot{x}=f(x,u)$, the corresponding function is clearly differentiable,
\item $x\in \mathbb{R}^2_+$ and $x_i=0$ implies $\dot{x}_i\leq 0$ for $i\in \{1,2\}$. Indeed,
\begin{itemize}
\item if $x_1=0$, $x_2\leq 0$, $\dot{x}_1=\frac{ux_2}{1+x_2}\leq 0$,
\item if $x_1\leq 0$, $x_2= 0$, $\dot{x}_2=\frac{ux_1}{1+x_1}\leq 0$.
\end{itemize}
\end{enumerate}
Hence, the system is positive.

\paragraph{Equilibriums:}~\\
The equilibriums are found by solving
$$
\begin{array}{rcl}
-x_1 +\frac{ux_2}{1+x_2} &=& 0\\
-x_2 +\frac{ux_1}{1+x_1} &=& 0
\end{array}
$$
We obtain two different cases:
\begin{itemize}
\item case 1: $\bar{u}\leq 1$ then there exists a unique equilibrium point in the positive orthant:
$$(\bar{x}_1,\bar{x}_2)=(0,0)$$
\item case 2: $\bar{u}>1$ then there exists two equilibrium points in the positive orthant:
$$
\begin{array}{rcl}
(\bar{x}^1_1,\bar{x}^1_2) &=& (0,0)\\
(\bar{x}^2_1,\bar{x}^2_2) &=& (\bar{u}-1,\bar{u}-1)
\end{array}
$$
\end{itemize}

\paragraph{Local stability of equilibriums:}~\\
The local stability of equilibriums can be verified using the first method of Lyapunov. The Jacobian of the system is given by :
$$
J_f=\begin{bmatrix}
-1 & \frac{u}{(1+x_2)^2}\\
\frac{u}{(1+x_1)^2} & -1
\end{bmatrix}
$$
For both equilibrium points we apply the method of Lyapunov:
\begin{enumerate}
\item $(\bar{x}^1_1,\bar{x}^1_2) = (0,0)$:
$$J_f(0,0) = \begin{bmatrix}
-1 & \bar{u}\\
\bar{u} & -1
\end{bmatrix} 
$$
Then, $spec(J_f(0,0))=\{-1+|\bar{u}|,-1-|\bar{u}|\}$, the equilibrium is thus surely stable when $|\bar{u}|<1$ and surely unstable when $|\bar{u}|>1$.
\item $(\bar{x}^2_1,\bar{x}^2_2) = (\bar{u}-1,\bar{u}-1)$ when $\bar{u}>1$:
$$J_f(\bar{u}-1,\bar{u}-1) = \begin{bmatrix}
-1 & \frac{1}{\bar{u}}\\
\frac{1}{\bar{u}} & -1
\end{bmatrix} 
$$
then $spec(J_f(\bar{u}-1,\bar{u}-1))=\{-1-\frac{1}{\bar{u}},-1+\frac{1}{\bar{u}}\}$, the equilibrium is thus surely stable (as $\bar{u}>1$).
\end{enumerate}

\paragraph{Global asymptotic stability of the origin: }~\\
For $0<\bar{u}<1$, we use the Lyapunov function $V(x_1,x_2)=\frac{1}{2}(x_1^2+x_2^2)$ to show that the origin is globally asymptotically stable in the positive orthant. We verify the conditions to ensure the asymptotic stability of the origin:
\begin{enumerate}
\item $V(0,0)=0<V(x_1,x_2)$ whenever $(x_1,x_2)\in\mathbb{R}^2_+\setminus\{(0,0)\}$
\item Derivative
$$
\begin{array}{rcl}
\dot{V}(x_1,x_2) &=& x_1\dot{x}_1+x_2\dot{x}_2\\
&=& -x_1^2-x_2^2+ux_1x_2(\frac{1}{1+x_1}+\frac{1}{1+x_2})\\
&<& -x_1^2-x_2^2+x_1x_2(\frac{1}{1+x_1}+\frac{1}{1+x_2})\\
&\leq& -x_1^2-x_2^2+2x_1x_2\\
&=& -(x_1-x_2)^2\leq 0
\end{array}
$$
\end{enumerate}
Hence the origin is globally asymptotically stable.

\subsection*{Exercise 9.5 : Chemical reactor}
Two reactions involving three chemicals ($X_1$, $X_2$ and $X_3$) are taking place in a continuous reactor (in liquid phase and with a constant volume). The dynamics of the reactor is described by the following state-space model:
$$
\begin{array}{rcl}
\dot{x}_1 &=& -x_1^2x_2-dx_1+du\\
\dot{x}_2 &=& x_1^2x_2-(d+k)x_2\\
\dot{x}_3 &=& 2kx_2-dx_3
\end{array}
$$
where $x_i$ denotes the concentration in chemical $i$.
\begin{enumerate}
\item what are the reactions involved (law of mass action),
\item find the equilibrium points in the positive orthant and for $\bar{u}>0$, give all equilibrium points for each value of $\bar{u}$ along with the conditions for their existence,
\item analyse the stability of each equilibrium using the first method of Lyapunov.
\end{enumerate}

\textbf{Answer}~\\
\paragraph{Reaction involved:}~\\
Clearly, the reaction kinetics involved in the state space model are
$$
\begin{array}{rcl}
r_1(x)&=&x_1^2x_2\\
r_2(x)&=&2kx_2
\end{array}
$$
Moreover, the analysis of the state-space model shows that the stoichiometric matrix is
$$
\begin{bmatrix}
-1 & 0\\
1 & -1\\
0 & 2
\end{bmatrix}
$$
Hence, the reactions are
$$
\begin{array}{rcl}
2X_1 + X_2 &\rightarrow& X_1 + 2X_2\\
X_2 &\rightarrow& 2X_3
\end{array}
$$

\paragraph{Equilibrium points: }~\\
The equilibrium points are found by solving the system:
$$
\begin{array}{rcl}
-\bar{x}_1^2\bar{x}_2-d\bar{x}_1+d\bar{u}&=&0\\
\bar{x}_1^2\bar{x}_2-(d+k)\bar{x}_2&=&0\\
2k\bar{x}_2-d\bar{x}_3&=&0
\end{array}
$$
The equilibrium points are the following:
\begin{enumerate}
\item $(\bar{x}_1^1,\bar{x}_2^1,\bar{x}_3^1)=(\bar{u},0,0)$ whenever $\bar{u}>0$,
\item $(\bar{x}_1^2,\bar{x}_2^2,\bar{x}_3^2)=(\sqrt{d+k},\frac{d(\bar{u}-\sqrt{d+k})}{d+k},\frac{2k(\bar{u}-\sqrt{d+k})}{d+k})$ whenever $\bar{u}>\sqrt{d+k}$ (equality leads to the first equilibrium point).
\end{enumerate}

\paragraph{Stability of equilibrium: }~\\
The Jacobian of the system is:
$$
J_f=\begin{bmatrix}
-2x_1x_2-d & -x_1^2 & 0\\
2x_1x_2 & x_1^2-(d+k) & 0\\
0 & 2k & -d
\end{bmatrix}
$$
To analyse the equilibrium, one can observe that $J_f$ has exactly the same spectrum as the matrix
$$ A(x_1,x_2,x_3)=\begin{bmatrix}
B & 0\\
0 & -d
\end{bmatrix}
$$
where
$$
B(x_1,x_2,x_3)=\begin{bmatrix}
-2x_1x_2-d & -x_1^2\\
2x_1x_2 & x_1^2-(d+k)
\end{bmatrix}
$$
Thus, the spectrum of the Jacobian is given by $spec(J_f(x_1,x_2,x_3))=\{-d\}\bigcup spec(B(x_1,x_2,x_3))$. As $d>0$ (input flow), the eigenvalue $-d$ is always negative real and we may focus on the spectrum of $B$ for each equilibrium point.

\begin{enumerate}
\item $(\bar{x}_1^1,\bar{x}_2^1,\bar{x}_3^1)=(\bar{u},0,0)$ whenever $\bar{u}>0$:
$$
B(\bar{x}_1^1,\bar{x}_2^1,\bar{x}_3^1)=\begin{bmatrix}
-d & -\bar{u}^2\\
0 & \bar{u}^2-(d+k)
\end{bmatrix}
$$
As $spec(B(\bar{x}_1^1,\bar{x}_2^1,\bar{x}_3^1))=\{-d,\bar{u}^2-(d+k)\}$, the equilibrium is asymptotically stable if $0<\bar{u}^2<d+k$.
\item $(\bar{x}_1^2,\bar{x}_2^2,\bar{x}_3^2)=(\sqrt{d+k},\frac{d(\bar{u}-\sqrt{d+k})}{d+k},\frac{2k(\bar{u}-\sqrt{d+k})}{d+k})$ whenever $\bar{u}>\sqrt{d+k}$:
$$
B(\bar{x}_1^1,\bar{x}_2^1,\bar{x}_3^1)=\begin{bmatrix}
-2\frac{d}{\sqrt{d+k}}(\bar{u}-\sqrt{d+k})-d & -(d+k)\\
2\frac{d}{\sqrt{d+k}}(\bar{u}-\sqrt{d+k}) & 0
\end{bmatrix}
$$
As $trace(B)=-2\frac{d}{\sqrt{d+k}}(\bar{u}-\sqrt{d+k})-d<0$ and $det(B)=2\frac{d}{\sqrt{d+k}}(\bar{u}-\sqrt{d+k})(d+k)>0$, the equilibrium point is asymptotically stable whenever $\bar{u}>\sqrt{d+k}$.
\end{enumerate}